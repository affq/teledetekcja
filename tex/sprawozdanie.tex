\documentclass[a4paper,12pt]{article}  % Klasa dokumentu
\usepackage[polish]{babel}              % Język dokumentu
\usepackage{amsmath, amssymb}           % Pakiety matematyczne
\usepackage{graphicx}                   % Do wstawiania grafiki
\usepackage{hyperref}                   % Linki w dokumencie
\usepackage{geometry}                   % Ustawienia marginesów
\usepackage{float}
\usepackage[T1]{fontenc}
\usepackage{subfigure}
\geometry{margin=1in}                   % Definicja marginesów
\usepackage{array}
\usepackage{caption}
\usepackage{subcaption}

% Tytuł i autor
\title{Obrazy kanałowe i wskaźniki spektralne}
\author{Adrian Fabisiewicz}
\date{\today}

\begin{document}

\maketitle  % Tworzenie tytułu

\section{Cel zadania}
Celem zadania jest zbadanie:
\begin{itemize}
    \item jak różne klasy obiektów odwzorowują się na obrazach w spektrum widzialnym (RGB) oraz w bliskiej podczerwieni (IR) 
    \item w jaki sposób kompozycje barwne mogą ułatwić interpretację klas obiektów na zdjęciach
    \item roli wskaźników spektralnych w interpretacji obrazów
\end{itemize}

\section{Dane do zadania}
Danymi do zadania były zdjęcia zarejestrowane w zakresie RGB oraz CIR z lat 2015 i 2023 oraz stworzona na ich podstawie poligonowa warstwa wektorowa, zawierająca budynki.

\section{Realizacja}
\subsection{Przedstawienie budynków w poszczególnych kanałach spektralnych}

\subsubsection{Budynek nr 1}

% \begin{figure}[H]
% \hfill
% \subfigure[NIR]{\includegraphics[width=3.5cm]{spektralne/nir_budynek0.png}}
% \hfill
% \subfigure[RED]{\includegraphics[width=3.5cm]{spektralne/red_budynek0.png}}
% \hfill
% \subfigure[GREEN]{\includegraphics[width=3.5cm]{spektralne/green_budynek0.png}}
% \hfill
% \subfigure[BLUE]{\includegraphics[width=3.5cm]{spektralne/blue_budynek0.png}}
% \hfill
% \end{figure}

\begin{figure}[H]
    \centering
    \begin{minipage}{0.24\textwidth}
        \centering
        \includegraphics[width=\linewidth]{spektralne/nir_budynek0.png}
        \caption*{NIR}
    \end{minipage}
    \begin{minipage}{0.24\textwidth}
        \centering
        \includegraphics[width=\linewidth]{spektralne/red_budynek0.png}
        \caption*{RED}
    \end{minipage}
    \begin{minipage}{0.24\textwidth}
        \centering
        \includegraphics[width=\linewidth]{spektralne/green_budynek0.png}
        \caption*{GREEN}
    \end{minipage}
    \begin{minipage}{0.24\textwidth}
        \centering
        \includegraphics[width=\linewidth]{spektralne/blue_budynek0.png}
        \caption*{BLUE}
    \end{minipage}
\end{figure}

\subsubsection{Budynek nr 2}

\begin{figure}[H]
    \centering
    \begin{minipage}{0.24\textwidth}
        \centering
        \includegraphics[width=\linewidth]{spektralne/nir_budynek3.png}
        \caption*{NIR}
    \end{minipage}
    \begin{minipage}{0.24\textwidth}
        \centering
        \includegraphics[width=\linewidth]{spektralne/red_budynek3.png}
        \caption*{RED}
    \end{minipage}
    \begin{minipage}{0.24\textwidth}
        \centering
        \includegraphics[width=\linewidth]{spektralne/green_budynek3.png}
        \caption*{GREEN}
    \end{minipage}
    \begin{minipage}{0.24\textwidth}
        \centering
        \includegraphics[width=\linewidth]{spektralne/blue_budynek3.png}
        \caption*{BLUE}
    \end{minipage}
\end{figure}

\subsubsection{Budynek nr 3}

\begin{figure}[H]
    \centering
    \begin{minipage}{0.24\textwidth}
        \centering
        \includegraphics[width=\linewidth]{spektralne/nir_budynek7.png}
        \caption*{NIR}
    \end{minipage}
    \begin{minipage}{0.24\textwidth}
        \centering
        \includegraphics[width=\linewidth]{spektralne/red_budynek7.png}
        \caption*{RED}
    \end{minipage}
    \begin{minipage}{0.24\textwidth}
        \centering
        \includegraphics[width=\linewidth]{spektralne/green_budynek7.png}
        \caption*{GREEN}
    \end{minipage}
    \begin{minipage}{0.24\textwidth}
        \centering
        \includegraphics[width=\linewidth]{spektralne/blue_budynek7.png}
        \caption*{BLUE}
    \end{minipage}
\end{figure}

\newpage
\subsection{Zestawienie wybranych statystyk numerycznych w poszczególnych kanałach obrazu.}

\begin{table}[h!]
\centering
\begin{tabular}{|c|c|c|c|}
\hline
\multicolumn{4}{|c|}{\textbf{Obiekt 1}} \\ \hline
\textbf{} & \textbf{min} & \textbf{max} & \textbf{mean} \\ \hline
\textbf{NIR} & 48 & 157 & 77\\ \hline
\textbf{RED} & 40 & 131 & 70\\ \hline
\textbf{GREEN} & 43 & 134 & 71\\ \hline
\textbf{BLUE} & 37 & 120 & 65\\ \hline
\end{tabular}
\end{table}

\begin{table}[h!]
\centering
\begin{tabular}{|c|c|c|c|}
\hline
\multicolumn{4}{|c|}{\textbf{Obiekt 2}} \\ \hline
\textbf{} & \textbf{min} & \textbf{max} & \textbf{mean} \\ \hline
\textbf{NIR} & 39 & 187 & 101\\ \hline
\textbf{RED} & 47 & 203 & 127\\ \hline
\textbf{GREEN} & 60 & 217 & 133\\ \hline
\textbf{BLUE} & 43 & 205 & 128\\ \hline
\end{tabular}
\end{table}

\begin{table}[h!]
\centering
\begin{tabular}{|c|c|c|c|}
\hline
\multicolumn{4}{|c|}{\textbf{Obiekt 3}} \\ \hline
\textbf{} & \textbf{min} & \textbf{max} & \textbf{mean} \\ \hline
\textbf{NIR} & 59 & 178 & 90\\ \hline
\textbf{RED} & 60 & 189 & 99\\ \hline
\textbf{GREEN} & 55 & 193 & 97\\ \hline
\textbf{BLUE} & 44 & 191 & 92\\ \hline
\end{tabular}
\end{table}

\subsection{Porównanie wyglądu obiektów na kompozycjach barwnych RGB, IrGB oraz dwóch innych wybranych kompozycjach.}

\subsubsection{Obiekt 1}
\begin{figure}[H]
    \centering
    \begin{minipage}{0.24\textwidth}
        \centering
        \includegraphics[width=\linewidth]{spektralne/rgb_budynek7.png}
        \caption*{RGB}
    \end{minipage}
    \begin{minipage}{0.24\textwidth}
        \centering
        \includegraphics[width=\linewidth]{spektralne/irgb_budynek7.png}
        \caption*{IrGB}
    \end{minipage}
    \begin{minipage}{0.24\textwidth}
        \centering
        \includegraphics[width=\linewidth]{spektralne/rgir_budynek7.png}
        \caption*{RGIr}
    \end{minipage}
    \begin{minipage}{0.24\textwidth}
        \centering
        \includegraphics[width=\linewidth]{spektralne/irrb_budynek7.png}
        \caption*{IrRB}
    \end{minipage}
\end{figure}

\subsubsection{Obiekt 2}
\begin{figure}[H]
    \centering
    \begin{minipage}{0.24\textwidth}
        \centering
        \includegraphics[width=\linewidth]{spektralne/rgb_budynek0.png}
        \caption*{RGB}
    \end{minipage}
    \begin{minipage}{0.24\textwidth}
        \centering
        \includegraphics[width=\linewidth]{spektralne/irgb_budynek0.png}
        \caption*{IrGB}
    \end{minipage}
    \begin{minipage}{0.24\textwidth}
        \centering
        \includegraphics[width=\linewidth]{spektralne/rgir_budynek0.png}
        \caption*{RGIr}
    \end{minipage}
    \begin{minipage}{0.24\textwidth}
        \centering
        \includegraphics[width=\linewidth]{spektralne/irrb_budynek0.png}
        \caption*{IrRB}
    \end{minipage}
\end{figure}

\subsubsection{Obiekt 3}
\begin{figure}[H]
    \centering
    \begin{minipage}{0.24\textwidth}
        \centering
        \includegraphics[width=\linewidth]{spektralne/rgb_budynek3.png}
        \caption*{RGB}
    \end{minipage}
    \begin{minipage}{0.24\textwidth}
        \centering
        \includegraphics[width=\linewidth]{spektralne/irgb_budynek3.png}
        \caption*{IrGB}
    \end{minipage}
    \begin{minipage}{0.24\textwidth}
        \centering
        \includegraphics[width=\linewidth]{spektralne/rgir_budynek3.png}
        \caption*{RGIr}
    \end{minipage}
    \begin{minipage}{0.24\textwidth}
        \centering
        \includegraphics[width=\linewidth]{spektralne/irrb_budynek3.png}
        \caption*{IrRB}
    \end{minipage}
\end{figure}


\subsection{Porównanie wyglądu obiektów na wskaźniku spektralnym NDVI oraz jednym innym wybranym wskaźniku.}
\subsubsection{Obiekt 1}
\begin{figure}[H]
    \centering
    \begin{minipage}{0.24\textwidth}
        \centering
        \includegraphics[width=\linewidth]{spektralne/ndvi_budynek7_2015.png}
        \caption*{NDVI 2015}
    \end{minipage}
    \begin{minipage}{0.24\textwidth}
        \centering
        \includegraphics[width=\linewidth]{spektralne/ndvi_budynek7_2023.png}
        \caption*{NDVI 2023}
    \end{minipage}
    \begin{minipage}{0.24\textwidth}
        \centering
        \includegraphics[width=\linewidth]{spektralne/bi_budynek7_2015.png}
        \caption*{BI 2015}
    \end{minipage}
    \begin{minipage}{0.24\textwidth}
        \centering
        \includegraphics[width=\linewidth]{spektralne/bi_budynek7_2023.png}
        \caption*{BI 2023}
    \end{minipage}
\end{figure}

\subsubsection{Obiekt 2}

\begin{figure}[H]
    \centering
    \begin{minipage}{0.24\textwidth}
        \centering
        \includegraphics[width=\linewidth]{spektralne/ndvi_budynek0_2015.png}
        \caption*{NDVI 2015}
    \end{minipage}
    \begin{minipage}{0.24\textwidth}
        \centering
        \includegraphics[width=\linewidth]{spektralne/ndvi_budynek0_2023.png}
        \caption*{NDVI 2023}
    \end{minipage}
    \begin{minipage}{0.24\textwidth}
        \centering
        \includegraphics[width=\linewidth]{spektralne/bi_budynek0_2015.png}
        \caption*{BI 2015}
    \end{minipage}
    \begin{minipage}{0.24\textwidth}
        \centering
        \includegraphics[width=\linewidth]{spektralne/bi_budynek0_2023.png}
        \caption*{BI 2023}
    \end{minipage}
\end{figure}

\subsubsection{Obiekt 3}

\begin{figure}[H]
    \centering
    \begin{minipage}{0.24\textwidth}
        \centering
        \includegraphics[width=\linewidth]{spektralne/ndvi_budynek3_2015.png}
        \caption*{NDVI 2015}
    \end{minipage}
    \begin{minipage}{0.24\textwidth}
        \centering
        \includegraphics[width=\linewidth]{spektralne/ndvi_budynek3_2023.png}
        \caption*{NDVI 2023}
    \end{minipage}
    \begin{minipage}{0.24\textwidth}
        \centering
        \includegraphics[width=\linewidth]{spektralne/bi_budynek3_2015.png}
        \caption*{BI 2015}
    \end{minipage}
    \begin{minipage}{0.24\textwidth}
        \centering
        \includegraphics[width=\linewidth]{spektralne/bi_budynek3_2023.png}
        \caption*{BI 2023}
    \end{minipage}
\end{figure}

\newpage
\subsection{Zestawienie dla wskaźników spektralnych wartości statystycznych wyznaczonych na danych z 2015 oraz 2023 roku}

\begin{table}[h!]
    \centering
    \begin{tabular}{|c|c|c|c|}
    \hline
    \multicolumn{4}{|c|}{\textbf{Obiekt 1}} \\ \hline
    \textbf{} & \textbf{min} & \textbf{max} & \textbf{mean} \\ \hline
    \textbf{NDVI 2015} & 0 & 255 & 55\\ \hline
    \textbf{NDVI 2023} & 0 & 255 & 108\\ \hline
    \textbf{.. 2015} & .. & .. & ..\\ \hline
    \textbf{.. 2023} & .. & .. & ..\\ \hline
    \end{tabular}
\end{table}

\begin{table}[h!]
    \centering
    \begin{tabular}{|c|c|c|c|}
    \hline
    \multicolumn{4}{|c|}{\textbf{Obiekt 2}} \\ \hline
    \textbf{} & \textbf{min} & \textbf{max} & \textbf{mean} \\ \hline
    \textbf{NDVI 2015} & 0 & 255 & 93\\ \hline
    \textbf{NDVI 2023} & 0 & 255 & 105\\ \hline
    \textbf{.. 2015} & .. & .. & ..\\ \hline
    \textbf{.. 2023} & .. & .. & ..\\ \hline
    \end{tabular}
\end{table}

\begin{table}[h!]
    \centering
    \begin{tabular}{|c|c|c|c|}
    \hline
    \multicolumn{4}{|c|}{\textbf{Obiekt 3}} \\ \hline
    \textbf{} & \textbf{min} & \textbf{max} & \textbf{mean} \\ \hline
    \textbf{NDVI 2015} & 0 & 255 & 66\\ \hline
    \textbf{NDVI 2023} & 0 & 255 & 100\\ \hline
    \textbf{.. 2015} & .. & .. & ..\\ \hline
    \textbf{.. 2023} & .. & .. & ..\\ \hline
    \end{tabular}
\end{table}

\section{Komentarz}

\end{document}
